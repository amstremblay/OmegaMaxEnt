%-------------------------------------------------------------------------------
%Dominic Bergeron
%-------------------------------------------------------------------------------

%D�finition du type (classe) de document
%-------------------------------------------------------------------------------
\documentclass[12pt,letterpaper,oneside,titlepage]{article}   
%-------------------------------------------------------------------------------

% Packages utiles
%-------------------------------------------------------------------------------
\usepackage{array}
%\usepackage[french]{babel}  % Utilise l'hyphenation francaise
\usepackage[dvips]{graphicx}    % Pour l'importation de graphiques EPS
\usepackage{amsmath}        % Utilise les fonctions mathematiques avancees
\usepackage{amssymb}
\usepackage{times}      % Utilise le font Times plutot que le font standard de LaTeX (Computer Modern). Simplifie les fichiers PDF.
%\usepackage{arial}
%\usepackage{boxedminipage} % Pour faire des blocs d'equations dans une boite
%\usepackage{lscape}        % Pour pouvoir mettre des pages en landscape (horizontal) 
%\usepackage{float}     % Pour faire des tableaux multipages
%\usepackage[ansinew]{inputenc}  % Pour que LaTeX reconnaisse les caracteres etendus (accents)
%\usepackage{epsfig}
%\usepackage{subfigure}
\usepackage{t1enc}      %c�sure des mots accentu�s
\usepackage{setspace}
%\usepackage[pdftex=true]{hyperref}
\usepackage{hyperref}
%\usepackage{feynmp}
%-------------------------------------------------------------------------------


%D�finition des marges et espacements
%-------------------------------------------------------------------------------

%\setlength{\parindent}{0in}

\setlength{\voffset}{-0.465in}
\setlength{\headheight}{12pt}
\setlength{\headsep}{0.424in}
\setlength{\topmargin}{0in}

\setlength{\hoffset}{0.2in}
\setlength{\oddsidemargin}{0in}

%%\setlength{\marginparwidth}{1in}
%%\setlength{\evensidemargin}{0.5in}

\setlength{\textwidth}{5.84in}
\setlength{\textheight}{8.5in}
\linespread{1.5}
\setlength{\parskip}{0.1in}
\setlength{\footskip}{0.3in}

%%\setlength{\subfigcapskip}{0pt}
%%\setlength{\subfigtopskip}{15pt}
%-------------------------------------------------------------------------------

%num�rote les �quations par num�ro de section
%-------------------------------------------------------------------------------
\numberwithin{equation}{section}    
%-------------------------------------------------------------------------------

%Nouvelles commandes pour l'affichage de symboles math�matiques
%-------------------------------------------------------------------------------
\newcommand\diverg{\nabla\cdot}
\renewcommand{\d}[2]{\frac{#1}{#2}}
\newcommand{\pd}{\partial}
\newcommand{\bra}[1]{\langle #1|}
\newcommand{\ket}[1]{| #1\rangle}
\newcommand{\mn}[1]{\left\langle #1 \right\rangle}
\newcommand{\bk}[2]{\left\langle #1 | #2 \right\rangle}
\newcommand{\derPrem}[2]{\frac{d#1}{d#2}}
\newcommand{\derSec}[2]{\frac{d^2#1}{d#2^2}}
%-------------------------------------------------------------------------------

%\title{Analytic continuation for spectral functions with non-definite sign}
%\author{Dominic Bergeron}
%\hyphenation{}

\begin{document}

\pagenumbering{arabic}

%\maketitle

%\clearpage

%\tableofcontents

%\clearpage

\section{Analytic continuation for spectral functions with non-definite sign}

Given a correlation function
\begin{equation}
C_{AB}(\tau)=-\mn{T_{\tau} A(\tau)B}\,,
\end{equation}
or its Fourier transform $C_{AB}(i\omega_n)$, analytic continuation of that function toward real frequencies using the maximum entropy method can only be done directly for spectral functions with the property 
\begin{equation}\label{eq:positivity_fermions_1}
Im[C_{AB}^R(\omega)]<0\,,
\end{equation}
if $A$ and $B$ are fermion operators, or
\begin{equation}\label{eq:positivity_bosons_1}
\frac{Im[C_{AB}^R(\omega)]}{\omega}<0\,,
\end{equation}
if $A$ and $B$ are boson operators. However, if $Im[C_{AB}^R(\omega)]$ does not satisfy that property, it can also be obtained \textit{indirectly}. Indeed, although more work is necessary, any such function can be obtained using a maximum entropy routine, provided other Matsubara correlation functions are also available \cite{Reymbaut:2015,Reymbaut_Gagnon_2017}. The procedure to to do so is the following:

First, we define
\begin{equation}
\begin{split}
O&=A+\mu B^\dagger\,,\\\
P&=A+i\nu B^\dagger\,.
\end{split}
\end{equation}
where $\mu$ and $\nu$ are real scalars such that $\mu B$ and $\nu B$ have the same dimension as $A$ and which values are assumed arbitrary in the present demonstration. Thus,
\begin{equation}
\begin{split}
C_{OO^\dagger}(\tau)%&=-\mn{T_{\tau} O(\tau)O^\dagger}\,,\\
&=-\mn{T_{\tau} (A+\mu B^\dagger)(\tau)(A+\mu B^\dagger)^\dagger}\\
&=-\mn{T_{\tau} (A+\mu B^\dagger)(\tau)(A^\dagger+\mu B)}\\
&=-\mn{T_{\tau} A(\tau)A^\dagger} -\mu\mn{T_{\tau}A(\tau)B}-\mu\mn{B^\dagger(\tau)A^\dagger}-\mu^2\mn{T_{\tau}B^\dagger(\tau)B}\\
C_{OO^\dagger}(\tau)&=C_{AA^\dagger}(\tau)+\mu C_{AB}(\tau)+\mu C_{B^\dagger A^\dagger}(\tau)+\mu^2 C_{B^\dagger B}(\tau)\,,
\end{split}\label{eq:def_COO}
\end{equation}
%\begin{equation}
%C_{OO^\dagger}(i\omega_n)=C_{AA^\dagger}(i\omega_n)+\mu C_{AB}(i\omega_n)+\mu C_{B^\dagger A^\dagger}(i\omega_n)+\mu^2 C_{B^\dagger B}(i\omega_n)\,,
%\end{equation}
and
\begin{equation}
\begin{split}
C_{PP^\dagger}(\tau)%&=-\mn{T_{\tau} P(\tau)P^\dagger}\,,\\
&=-\mn{T_{\tau} (A+i\nu B^\dagger)(\tau)(A+i\nu B^\dagger)^\dagger}\,,\\
&=-\mn{T_{\tau} (A+i\nu B^\dagger)(\tau)(A^\dagger-i\nu B)}\,,\\
&=-\mn{T_{\tau} A(\tau)A^\dagger} +i\nu\mn{T_{\tau}A(\tau)B}-i\nu\mn{B^\dagger(\tau)A^\dagger}-\nu^2\mn{T_{\tau}B^\dagger(\tau)B}\\
C_{PP^\dagger}(\tau)&=C_{AA^\dagger}(\tau) -i\nu C_{AB}(\tau)+i\nu C_{B^\dagger A^\dagger}(\tau)+\nu^2C_{B^\dagger B}(\tau)\,.
\end{split}\label{eq:def_CPP}
\end{equation}
Now, if we define
\begin{equation}
\begin{split}
%R(\tau)&=C_{OO^\dagger}(\tau)+\mn{T_{\tau} A(\tau)A^\dagger}+\mu^2\mn{T_{\tau}B^\dagger(\tau)B}\\
R(\tau)&=C_{OO^\dagger}(\tau)-C_{AA^\dagger}(\tau)-\mu^2C_{B^\dagger B}(\tau)\\
&=\mu\left[C_{AB}(\tau)+C_{B^\dagger A^\dagger}(\tau)\right]
\end{split}\label{eq:R_tau}
\end{equation}
and
\begin{equation}
\begin{split}
%S(\tau)&=C_{PP^\dagger}(\tau)+\mn{T_{\tau} A(\tau)A^\dagger}+\nu^2\mn{T_{\tau}B^\dagger(\tau)B}\\
S(\tau)&=C_{PP^\dagger}(\tau)-C_{AA^\dagger}(\tau)-\nu^2C_{B^\dagger B}(\tau)\\
&=-i\nu\left[C_{AB}(\tau)-C_{B^\dagger A^\dagger}(\tau)\right]\,,
\end{split}\label{eq:S_tau}
\end{equation}
we have
\begin{equation}
C_{AB}(\tau)=\frac{1}{2}\left[\frac{1}{\mu}R(\tau)+i\frac{1}{\nu}S(\tau)\right]\,,
\end{equation}
or
\begin{equation}
C_{AB}(i\omega_n)=\frac{1}{2}\left[\frac{1}{\mu}R(i\omega_n)+i\frac{1}{\nu}S(i\omega_n)\right]\,,
\end{equation}
or
\begin{equation}
C_{AB}^R(\omega)=\frac{1}{2}\left[\frac{1}{\mu}R^R(\omega)+i\frac{1}{\nu}S^R(\omega)\right]\,.
\end{equation}
Therefore,
\begin{equation}\label{eq:Im_C_AB}
Im[C_{AB}^R(\omega)]=\frac{1}{2}\left[\frac{1}{\mu}Im[R^R(\omega)]+\frac{1}{\nu}Re[S^R(\omega)]\right]\,,
\end{equation}
where, from \eqref{eq:R_tau} and \eqref{eq:S_tau},
\begin{equation}\label{eq:R_R_omega}
R^R(\omega)=C_{OO^\dagger}^R(\omega)-C_{AA^\dagger}^R(\omega)-\mu^2C_{B^\dagger B}^R(\omega)\\
\end{equation}
and
\begin{equation}\label{eq:S_R_omega}
S^R(\omega)=C_{PP^\dagger}^R(\omega)-C_{AA^\dagger}^R(\omega)-\nu^2C_{B^\dagger B}^R(\omega)\,.
\end{equation}

Now, the right-hand sides of \eqref{eq:R_tau} and \eqref{eq:S_tau} contain only correlation functions that satisfy property \eqref{eq:positivity_fermions_1} or \eqref{eq:positivity_bosons_1} (see section \ref{sec:definite_sign_sf} for the proof)  and can thus be treated directly with a maximum entropy (MaxEnt) routine. To obtain $Im[R^R(\omega)]$, it suffices to obtain $Im[C_{OO^\dagger}^R(\omega)]$, $Im[C_{AA^\dagger}(\omega)]$ and $Im[C_{B^\dagger B}(\omega)]$ separately with the MaxEnt routine and then use \eqref{eq:R_R_omega}. If the correlation function has the property $C_{AB}(\tau)=C_{B^\dagger A^\dagger}(\tau)$, then $S(\tau)=0$ and the problem is solved. Otherwise, to obtain $Re[S^R(\omega)]$ in \eqref{eq:Im_C_AB} using \eqref{eq:S_R_omega}, we need $Re[C_{PP^\dagger}(\omega)]$, $Re[C_{AA^\dagger}(\omega)]$ and $Re[C_{B^\dagger B}(\omega)]$. Those functions can be computed using the Kramers-Kr\"onig relation:
\begin{equation}
Re[C_{AB}^R(\omega)]=\mathcal{P}\int \frac{d\omega'}{\pi} \frac{Im[C_{AB}^R(\omega)]}{\omega'-\omega}\,,
\end{equation}
once the imaginary parts have been computed with the MaxEnt routine. Therefore, all functions necessary to obtain $Im[C_{AB}^R(\omega)]$ can be obtained with a maximum entropy routine and a Kramers-Kr\"onig relation.

To summarize, the steps to obtain $Im[C_{AB}^R(\omega)]$ are
\begin{enumerate}
\item Compute $C_{AB}(\tau)$, $C_{B^\dagger A^\dagger}(\tau)$, $C_{AA^\dagger}(\tau)$ and $C_{B^\dagger B}(\tau)$, or their Matsubara frequency counterparts
\item Define the parameters $\mu$ and $\nu$ and compute $C_{OO^\dagger}(\tau)$, given by \eqref{eq:def_COO} and $C_{PP^\dagger}(\tau)$, given by \eqref{eq:def_CPP}, or their Matsubara frequency counterparts
\item Use a maximum entropy routine to compute $Im[C_{OO^\dagger}^R(\omega)]$, $Im[C_{PP^\dagger}(\omega)]$ (if $C_{AB}(\tau)\neq C_{B^\dagger A^\dagger}(\tau)$), $Im[C_{AA^\dagger}(\omega)]$ and $Im[C_{B^\dagger B}(\omega)]$
\item If $C_{AB}(\tau)\neq C_{B^\dagger A^\dagger}(\tau)$, use the appropriate Kramers-Kr\"onig relation to obtain $Re[C_{PP^\dagger}(\omega)]$, $Re[C_{AA^\dagger}(\omega)]$ and $Re[C_{B^\dagger B}(\omega)]$
\item Compute $Im[C_{AB}^R(\omega)]$ with \eqref{eq:Im_C_AB}, \eqref{eq:R_R_omega} and \eqref{eq:S_R_omega}
\end{enumerate}

In practice, the stability of the result will depend on the values of $\mu$ and $\nu$. Those values must therefore be varied to find a region of the $(\mu,\nu)$ space where the result is stable \cite{Reymbaut:2015}.

This method was originally published in Ref. \cite{Reymbaut:2015}, including an example of application on the calculation of the ``anomalous'' part of the Green function in superconductivity. Another example of application on the calculation of thermoelectric (Seebeck) response functions can be found in Ref. \cite{Reymbaut_Gagnon_2017}.

To summarize, the steps to obtain $Im[C_{AB}^R(\omega)]$ are
\begin{enumerate}
\item Compute $C_{AB}(\tau)$, $C_{B^\dagger A^\dagger}(\tau)$, $C_{AA^\dagger}(\tau)$ and $C_{B^\dagger B}(\tau)$
\item Define $\mu$ and $\nu$ and compute $C_{OO^\dagger}(\tau)$, $C_{PP^\dagger}(\tau)$
\item Use MaxEnt to compute $Im[C_{OO^\dagger}^R(\omega)]$, $Im[C_{PP^\dagger}^R(\omega)]$, $Im[C_{AA^\dagger}^R(\omega)]$ and $Im[C_{B^\dagger B}^R(\omega)]$, then $Im[R^R(\omega)]$
\item Use Kramers-Kr\"onig to obtain $Re[C_{PP^\dagger}(\omega)]$, $Re[C_{AA^\dagger}(\omega)]$ and $Re[C_{B^\dagger B}(\omega)]$, then $Re[S^R(\omega)]$ 
\end{enumerate}


\section{Spectral function with definite sign}\label{sec:definite_sign_sf}

The spectral function $Im[C_{AA^\dagger}^R(\omega)]$ associated with a correlation function of the form
\begin{equation}
C_{AA^\dagger}(\tau)=-\mn{T_{\tau} A(\tau)A^\dagger}
\end{equation}
has the property
\begin{equation}\label{eq:positivity_fermions}
Im[C_{AA^\dagger}^R(\omega)]<0\,,
\end{equation}
if $A$ is a fermion operator, or
\begin{equation}\label{eq:positivity_bosons}
\frac{Im[C_{AA^\dagger}^R(\omega)]}{\omega}<0\,,
\end{equation}
if $A$ is a boson operator.

\textit{Proof:}

Let us assume $\tau>0$. If $\ket{i}$ are eigenstates of $H$, then
\begin{equation}
\begin{split}
C_{AA^\dagger}(\tau)&=-\frac{1}{Z}\sum_i\bra{i}e^{-\beta H}e^{\tau H}Ae^{-\tau H}A^\dagger\ket{i}\\
&=-\frac{1}{Z}\sum_{ij}\bra{i}e^{-\beta H}e^{\tau H}Ae^{-\tau H}\ket{j}\bra{j}A^\dagger\ket{i}\\
&=-\frac{1}{Z}\sum_{ij}e^{-\beta E_i}e^{\tau \left(E_i-E_j\right)}\bra{i}A\ket{j}\bra{j}A^\dagger\ket{i}\\
&=-\frac{1}{Z}\sum_{ij}e^{-\beta E_i}e^{\tau \left(E_i-E_j\right)}\left|\bra{i}A\ket{j}\right|^2\,,
\end{split}
\end{equation}
where
\begin{equation}
Z=\sum_i\bra{i}e^{-\beta H}\ket{i}=\sum_ie^{-\beta E_i}\,.
\end{equation}

The Fourier transform of $C_{AA^\dagger}(\tau)$ is
\begin{equation}
\begin{split}
C_{AA^\dagger}(i\omega_n)&=-\frac{1}{Z}\sum_{ij}e^{-\beta E_i}\left|\bra{i}A\ket{j}\right|^2\int_{0}^\beta d\tau e^{i\omega_n\tau}e^{\tau \left(E_i-E_j\right)}\\
&=-\frac{1}{Z}\sum_{ij}\frac{\left|\bra{i}A\ket{j}\right|^2}{i\omega_n+E_i-E_j}e^{-\beta E_i}\left(e^{i\omega_n\beta}e^{\beta\left(E_i-E_j\right)}-1\right)\\
&=-\frac{1}{Z}\sum_{ij}\frac{\left|\bra{i}A\ket{j}\right|^2}{i\omega_n+E_i-E_j}\left(\pm e^{-\beta E_j}-e^{-\beta E_i}\right)\\
\end{split}
\end{equation}
where $+$ is for bosons and $-$ is for fermions. Now the retarded correlation function is
\begin{equation}
C_{AA^\dagger}^R(\omega)=-\lim_{\eta\rightarrow 0^+}\frac{1}{Z}\sum_{ij}\frac{\left|\bra{i}A\ket{j}\right|^2}{\omega+i\eta+E_i-E_j}\left(\pm e^{-\beta E_j}-e^{-\beta E_i}\right)\\
\end{equation}
and thus,
\begin{equation}
Im[C_{AA^\dagger}^R(\omega)]=\frac{\pi}{Z}\sum_{ij}\left|\bra{i}A\ket{j}\right|^2\left(\pm e^{-\beta E_j}-e^{-\beta E_i}\right)\delta(\omega+E_i-E_j)\,.
\end{equation}

Therefore,
\begin{equation}
Im[C_{AA^\dagger}^R(\omega)]<0\,,\qquad fermions\\\,,
\end{equation}
\begin{equation}
Im[C_{AA^\dagger}^R(\omega)]
\begin{cases}
<0\,,\quad\omega>0\\
>0\,,\quad\omega<0\\
\end{cases},\qquad bosons
\end{equation}
i.e.
\begin{equation}
\frac{Im[C_{AA^\dagger}^R(\omega)]}{\omega}<0\,,\qquad bosons.
\end{equation}

\bibliographystyle{plain}
\bibliography{tout2013}

\end{document}

